\documentclass{article}
% 这里是导言区, 比如,我们通常在导言区设置页面大小、页眉页脚样式、章节标题样式等等。
\usepackage{xeCJK}
\setCJKmainfont{STSong}
\usepackage{graphicx}
% ##########################
% 实现首行缩进
\usepackage{indentfirst}
\setlength{\parindent}{2em}
% ##########################


\title{螺旋态器件在神经上的应用}
\author{郑浩}
\date{\today}




\begin{document}
	\maketitle
	\tableofcontents
	
	
	\section{中文摘要}
		对神经的操作和记录在科研和临床上是重要的,是有很大需求的。
		我们开发了一款基于形状记忆柔性材料的器件,整合不同组件可达到不同功能,并在生物体上测试。
		测试结果告诉我们器件能够对神经信号进行收集,能够给神经电刺激和光遗传刺激,同时能够通过精确的光遗传刺激激活神经的不同区域对下游肌肉有针对性地激活。
		通过完善器件的驱动设备,可使器件满足急性和慢性的多种使用需求。
	\section{英文摘要}
		The operation and recording of nerves is important in scientific research and clinical practice, and there is a great demand.
		We have developed a device based on shape memory flexible materials that integrates different components to achieve different functions and is tested on living organisms.
		The test results tell us that the device can collect neural signals, can give nerve electrical stimulation and optogenetic stimulation, and at the same time can activate different areas of the nerve through targeted optogenetic stimulation to target downstream muscles.
		By perfecting the device's driving equipment, the device can meet a variety of acute and chronic use needs.
	\section{引言}
		(描述清楚对神经的操作是重要的,需求在科研上和临床上都存在)
	\section{正文}
	\section{参考文献}
	\section{后记和致谢}
	
	
	\subsection{格式}
	\paragraph{格式}
	\subparagraph{格式}
	
	\includegraphics[width = .8\textwidth]{WechatIMG65.jpeg}
	%使用
	
	测试哦测试
	
\end{document}
